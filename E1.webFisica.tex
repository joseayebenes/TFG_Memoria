\chapterbegin{Web f\'{i}sica}
\label{chp:webfisica}
\minitoc
	
	\begin{wrapfigure}{r}{0.4\textwidth}
		\begin{center}
			\includegraphics[width=0.2\textwidth]{graphs/phyWeb-logo}
		\end{center}
		\caption{Logo de la web f\'{i}sica}
	\end{wrapfigure}
	
	El primer concepto que hay que conocer para afrontar el proyecto es el de Web F�sica. La Web F�sica es un t�rmino que describe el proceso de presentar objetos cotidianos en internet\cite{enablingInternetThings}. Este enfoque ofrece a los usuarios m�viles la posibilidad de gestionar sus tareas diarias en el uso de objetos cotidianos. Los objetos comienzan a ser inteligentes y remotamente controlables. Este modelo permite a los usuarios m�viles navegar y controlar los objetos f�sicos que rodean al dispositivo m�vil. Adem�s, esto ayuda al desempe�o de tareas diarias dependiendo de los objetos cercanos \cite{phisicalWebSmartCities}.\\
	
	Podemos mencionar en este contexto a los conocidos c�digos \ac{QR}, los c�digos \ac{QR} son un c�digo de barras en dos dimensiones, a menudo utilizados para mapear \ac{URL} con objetos f�sicos \cite{recognitionQRCodeMobile}.\\
	
	Las etiquetas inal�mbricas son uno de los enfoques m�s utilizados para el marcado de objetos f�sicos. Las etiquetas inal�mbricas pueden soportar protocolos como \ac{BLE} y \textit{WiFi}. Los protocolos mencionados son soportados por la mayor�a de los tel�fonos m�viles modernos.\\
	
	
	\begin{figure}[h]
		\centering
		\def\svgwidth{\columnwidth}
		\resizebox{0.7\textwidth}{!}{\input{graphs/dibujo.pdf_tex}}
		\caption{Ejemplos de comunicaci\'{o}n cercana}
	\end{figure}
	
	En la Web F�sica, personas, lugares y objetos tienen p�ginas web que proveen informaci�n y mecanismos de interacci�n. Sin embargo, es la amplitud y la profundidad de la pila que rodea a la web, que hacen de esta una atractiva visi�n para la evoluci�n del \ac{IOT}.\cite{enablingInternetThings}\\
	
	Las p�ginas web son una fant�stica tecnolog�a para interacci�n \ac{H2M}, pero muchos de los casos de uso del \ac{IOT} son interacciones \ac{M2M}. Los formatos de datos usados por \url{Schema.org} y otros, permiten a los agentes de usuario y servicios en la nube analizar los datos para eventos, organizaciones, personas, lugares, productos y as� sucesivamente, acutando sobre ellos de forma interactiva y proactiva.  \cite{enablingInternetThings}\\
	
	Uno de los primeros proyectos en fomentar esta idea fue \ac{HP} con \textit{Cooltown}, que usaba balizas infrarrojas para transmitir URLs. M�s recientemente, \ac{BLE} proporciona una similar baliza de bajo consumo que puede emitir \ac{URL} en paquetes peri�dicos (\url{www.uribeacon.org}). \cite{enablingInternetThings}\\
	
	
\chapterend{}