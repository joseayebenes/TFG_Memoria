%%%%%%%%%%%%%%%%%%%%%%%%%%%%%%%%%%
% P�gina de resumen del proyecto %
%%%%%%%%%%%%%%%%%%%%%%%%%%%%%%%%%%

\thispagestyle{empty}
\begin{center}
	\Large \sffamily
	Universidad de M�laga\\
	Escuela T�cnica Superior de Ingenier�a de\\
	Telecomunicaci�n
\end{center}

\bigskip

\begin{center}
	\Huge\scshape
	\pfctitlename
\end{center}

\bigskip

\begin{center}
	\textbf{Autor}\\
	\textsf{\pfcauthorname}
\end{center}

\medskip

\begin{center}
	\textbf{Tutores}\\
	\textsf{\pfctutorname}
\end{center}

\bigskip

\begin{minipage}{\textwidth}
\textbf{Departamento:} Tecnolog�a Electr�nica (DTE)

\medskip

\textbf{Titulaci�n:} Grado en Ingenier�a de Tecnolog�as de Telecomunicaci�n

\medskip

\textbf{Palabras clave:} IOT, web f�sica, web, red, sensores, inal�mbrica, internet, servidor


\medskip

\textbf{Resumen:}\\

Este trabajo de fin de grado implementa la integraci�n de una red inal�mbrica en la banda de 868 MHz utilizando el microcontrolador CC1350 que permite la interacci�n con dispositivos m�viles usando tecnolog�a web, a esto se le conoce como web f�sica.\\

Este proyecto ha implicado el desarrollo de la red inal�mbrica y un servicio web donde es posible monitorizar y gestionar la red. El servicio web consta de un Back-end usando NodeJS y un Front-end utilizando AngularJS.\\

Con el objetivo de demostrar el funcionamiento de la plataforma se ha realizado una aplicaci�n con dos nodos diferentes uno que env�a paquetes de la web f�sica y otro que simular�a que es un aparcamiento de bicicletas p�blicas.

\end{minipage}

\blankpage