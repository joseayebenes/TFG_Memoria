%%%%%%%%%%%%%%%%%%%%%%%%%%%%%%%%%%
% P�gina de resumen del proyecto %
%%%%%%%%%%%%%%%%%%%%%%%%%%%%%%%%%%

\thispagestyle{empty}
\begin{center}
	\Large \sffamily
	Universidad de M�laga\\
	Escuela T�cnica Superior de Ingenier�a de\\
	Telecomunicaci�n
\end{center}

\bigskip

\begin{center}
	\Huge\scshape
	\pfctitlename
\end{center}

\bigskip

\begin{center}
	\textbf{REALIZADO POR}\\
	\textsf{\pfcauthorname}
\end{center}

\medskip

\begin{center}
	\textbf{DIRIGIDO POR}\\
	\textsf{\pfctutorname}
\end{center}

\bigskip

\begin{minipage}{\textwidth}
\textbf{Dpto. de:} Tecnolog�a Electr�nica (DET)

\medskip

\textbf{Palabras clave:} IOT, web, red, sensores, inal�mbrica, f�sica, objetos, internet

\medskip

\textbf{Titulaci�n:} Grado en Ingenier�a de Tecnolog�as de Telecomunicaci�n

\medskip

\textbf{Resumen:}
	
Cuando la �nica forma de interactuar con alg�n dispositivo es a trav�s de una aplicaci�n especialidad el manejo de estos dispositivos se complica. De este problema nace la web f�sica que se enfoca en eliminar las aplicaciones y volver a lo fundamental de la web: la URL.\\

Este trabajo de fin de grado ofrece una plataforma donde sea posible la comunicaci�n desde un tel�fono m�vil a un dispositivo que �nicamente te ha enviado una URL. Para ello se estable una red inal�mbrica a 868 MHz usando el microcontrolador CC1350 donde hay conexi�n desde un servidor a los nodos.\\



\begin{center} M�laga, \today\end{center}
\end{minipage}

\blankpage