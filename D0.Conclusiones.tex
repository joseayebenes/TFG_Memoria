\chapterbeginx{Conclusiones}

Despu�s del desarrollo del proyecto, es pertinente hacer una valoraci�n final del mismo, respecto a los resultados obtenidos, las expectativas o el resultado de la experiencia acumulada.\\

Este \ac{TFG} nace con el objetivo de ser un trabajo multidisciplinar que abarque gran parte de los conocimientos adquiridos durante una titulaci�n generalista, como son las comunicaciones, telem�tica y electr�nica. Desde este punto se hizo un an�lisis de diferentes tecnolog�as recientes y se seleccionaron dos como principales: el concepto de ``Web f�sica'' y el microntrolador ``CC1350''. De esta forma surge el objetivo principal de este \ac{TFG} que es la integraci�n de estas dos tecnolog�as en una plataforma que pudiera utilizarse como una soluci�n al problema de la interacci�n a trav�s de la red.\\

En este TFG se ha conseguido con �xito la integraci�n de ambas tecnolog�as en el marco de la web f�sica desde la capa f�sica hasta las capas m�s altas. \\


Durante el desarrollo de este \ac{TFG} surgieron problemas debidos principalmente al poco tiempo que llevaba el microcontrolador en el mercado que dificultaba la b�squeda de informaci�n. Con el tiempo fue surgiendo m�s documentaci�n que hizo cambiar la linea de dise�o del proyecto hasta llegar a la que plasma este \ac{TFG}.\\

Para este proyecto se ha utilizado Github que es un servicio para alojar proyectos utilizando el sistema de control de versiones GIT. Al inicio del proyecto se crearon cuatro repositorios diferentes donde se ha guardado la mayor parte de la informaci�n del \ac{TFG}. Los repositorios son enumerados a continuaci�n:
\begin{description}
	\item[TFG\_memoria] (https://github.com/joseayebenes/TFG\_Memoria) Almacena la informaci�n sobre esta memoria escrita en \LaTeX{}.
	\item[TFG\_HostCollector] (https://github.com/joseayebenes/TFG\_HostCollector) Este repositorio contiene el proyecto del concentrador o nodo central de la red.
	\item[TFG\_Sensor] (https://github.com/joseayebenes/TFG\_Sensor) Contiene el c�digo que ejecutan los nodos de la red.
	\item[TFG\_Gateway] (https://github.com/joseayebenes/TFG\_Gateway) En este se guarda el proyecto del servidor.
\end{description}

Finalmente, se han visto concluidos satisfactoriamente los principales objetivos de este proyecto, que han permitido un mejor conocimiento de las tecnolog�as utilizadas y de la metodolog�a de trabajo para llevar un proyecto desde el concepto hasta el prototipo.

\chapterend{}
