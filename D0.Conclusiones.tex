\chapterbeginx{Conclusiones}

Despu�s del desarrollo del proyecto, es pertinente hacer una valoraci�n final del mismo, respecto a los resultados obtenidos, las expectativas o el resultado de la experiencia acumulada.\\

Este \ac{TFG} nace con el objetivo de ser un trabajo multidisciplinar que abarque gran parte de los conocimientos adquiridos durante una titulaci�n generalista, como son las comunicaciones, telem�tica y electr�nica. Desde este punto se hizo un an�lisis de diferentes tecnolog�as recientes y se seleccionaron dos como principales: el concepto de ``Web f�sica'' y el microntrolador ``CC1350''. De esta forma surge el objetivo principal de este \ac{TFG} que es la integraci�n de estas dos tecnolog�as en una plataforma que pudiera utilizarse como una soluci�n al problema de la interacci�n humano-m�quina.\\

Durante el desarrollo de este \ac{TFG} surgieron problemas debidos principalmente al peque�o tiempo que llevaba el microcontrolador en el mercado que dificultaba la b�squeda de informaci�n. Con el tiempo fue surgiendo m�s documentaci�n que hizo cambiar la linea de dise�o del proyecto hasta llegar a la que plasma este \ac{TFG}.\\


Finalmente, se han visto concluidos satisfactoriamente los principales objetivos de este proyecto, que han permitido un mejor conocimiento de las tecnolog�as utilizadas y de la metodolog�a de trabajo para llevar un proyecto desde concepto hasta prototipo, pasando por las fases de investigaci�n,   de un producto como ha sido este trabajo de fin de grado.

\chapterend
\chapterbeginx{L�neas futuras}

Durante el desarrollo del \ac{TFG}  se ha observado un crecimiento en el uso de las tecnolog�as que implementa este proyecto, lo que vaticina un futuro donde el concepto de web f�sica sea familiar a la poblaci�n. De este crecimiento se extraen lineas futuras de desarrollo en este �mbito como son:

\begin{itemize}
	\item Mejora del alcance analizando las diferentes alternativas que ofrece TI 15.4-Stack.
	\item Crear un nodo que emule una m�quina expendedora, donde el usuario pueda realizar su compra a trav�s de la web f�sica.
	\item Mejora de la duraci�n de la bater�a optimizando el c�digo y a�adiendo tecnolog�a de recolecci�n de energ�a.
	\item Creaci�n de placas de circuito impreso para nodo y concentrador que permita reducir el tama�o.
\end{itemize}


\begin{flushright}
{\large \pfcauthorname}\nli
\today
\end{flushright}
	
\chapterend