
\chapterbegin{Objetivos}
\label{chp:objetivo}

La web f�sica es un t�rmino que describe la forma de comunicar cualquier objeto f�sico con la web. A partir este enfoque, es posible navegar y controlar objetos en el mundo a trav�s de dispositivos m�viles. Esto ofrece a los usuarios la forma de realizar sus tareas diarias utilizando los objetos de su entorno. Para utilizar este enfoque, lo primero es seleccionar la manera con la que el objeto se comunicar� con el usuario, tal como c�digos QR o etiquetas RFID. De los diferentes enfoques que tiene la web f�sica, el que se va a tratar en este proyecto es el basado en proximidad inal�mbrica.\\

El modelo que plantea la web f�sica, los objetos pueden necesitan un canal por donde enviar y recibir datos y el problema se agrava cuando los objetos est�n distribuidos y no tienen cerca un punto de acceso a internet o tienen que funcionar con bater�as. En estos casos, buscar una soluci�n de comunicaci�n inal�mbrica de gran alcance y bajo consumo es prioritario.\\

Con estos conceptos en mente, el presente trabajo de fin de grado va a abordar el concepto de web f�sica, implementando una red inal�mbrica que permita comunicarse con los objetos con el fin de ser una red vers�til. Finalmente, esta tecnolog�a se aplicar� en un caso concreto de uso.


\chapterend{}