
\chapterbegin{Metodolog�a}
\label{chp:metod}

Para la correcta realizaci�n de este \ac{TFG} se han utilizado diversas herramientas y metodolog�as.\\

Cuando se aborda un proyecto de investigaci�n, la primera fase consiste en investigar y aprender sobre la materia. Para esto se han utilizado tres portales principalmente: Google Acad�mico, IEEE Xplore\R y la web de Texas Instruments. De estos portales se ha extra�do la informaci�n necesaria para el desarrollo de este proyecto.\\

Los proyectos software normalmente necesitan un control de versiones, para poder trazar qu� cambios se han realizado en el proyecto y poder volver a versiones anteriores si fuera necesario. Para este proyecto se ha utilizado Github que es un servicio para alojar proyectos utilizando el sistema de control de versiones GIT. Al inicio del proyecto se crearon cuatro repositorios diferentes donde se ha guardado la mayor parte de la informaci�n del \ac{TFG}. Los repositorios son enumerados a continuaci�n:

\begin{description}
	\item[TFG\_memoria] (https://github.com/joseayebenes/TFG\_Memoria) Almacena la informaci�n sobre esta memoria escrita en \LaTeX{}.
	\item[TFG\_HostCollector] (https://github.com/joseayebenes/TFG\_HostCollector) Este repositorio contiene el proyecto del concentrador o nodo central de la red.
	\item[TFG\_Sensor] (https://github.com/joseayebenes/TFG\_Sensor) Contiene el c�digo que ejecutan los nodos de la red.
	\item[TFG\_Gateway] (https://github.com/joseayebenes/TFG\_Gateway) En este se guarda el proyecto del servidor.
\end{description}

Cada uno de los repositorios alberga partes diferentes del proyecto y cada una de estas partes han tenido herramientas y metodolog�as diferentes.

\sectionx{Memoria} 

La memoria es la parte final de un proyecto, es donde se resumen la investigaci�n y se presentan los resultados. Para su realizaci�n se ha utilizado \LaTeX{}, que es un sistema de composici�n de textos y el entorno TeXstudio para facilitar el trabajo y compilaci�n del documento.Adem�s se han utilizado herramientas c�mo StarUML o  Adobe Illustrator para la creaci�n de los diagramas e im�genes.\\

\sectionx{Concentrador}

Esta parte est� destinada a ser ejecuta en un entorno Linux, por lo que ha sido programado en C++, usando el editor de textos SublimeText para la edici�n de los archivos de c�digo.\\

Uno de los inconvenientes al realizar esta parte era trabajar en un sistema operativo distinto al instalado en el ordenador, por lo que en las primeras fases del proyecto se utiliz� una m�quina virtual con el sistema operativo Ubuntu en VirtualBox, aunque la falta de recursos hac�a lento el desarrollo. C�mo soluci�n, se comenz� a trabajar utilizando la Raspberry Pi que finalmente se incluir�a en el proyecto creando unos \textit{scripts} que automatizaban la tarea de descargar el c�digo de GitHub, compilar y ejecutar. De esta forma se puede programar en cualquier sistema operativo y al subir los cambios a GitHub, la Raspberry Pi se los descargaba, compilaba y ejecutaba.\\

\sectionx{Servidor}

El servidor est� basado en Node.js\R, que es un entorno de ejecuci�n para JavaScript construido con el motor de JavaScript V8 de Chrome. El c�digo en JavaScript del servidor se realiz� utilizando SublimeText.\\

La parte del \textit{FrontEnd}, es decir, la parte con la que el usuario interact�a, est� basada en el \textit{FrameWork} AngularJS, que es una \textit{FrameWork} de JavaScript de c�digo abierto basado en el patr�n de dise�o \ac{MVC}.\\

En la fase final, el servidor se aloj� en un servidor dedicado de Amazon WebServices basado en Ubuntu Server, donde se crearon \textit{scripts} similares a los del concentrador para automatizar los despliegues de los cambios realizados.

\sectionx{Nodo}

El nodo est� programado en C usando el IDE CodeComposerStudio y librer�as del fabricante.\\

Para facilitar el aprendizaje del funcionamiento del c�digo tanto del Nodo como del concentrador, se ha documentado usando la herramienta Doxygen. Se podr� encontrar en el CD-ROM que acompa�a a este \ac{TFG} o en formato Html en los repositorios correspondientes.

\chapterend