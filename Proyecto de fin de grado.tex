% Pre�mbulo del documento.
%-----------------------------------------------------------------%
% Clase de documento: libro
\documentclass[12pt,a4paper]{book} % article, report, book.

% Pre�mbulo: paquetes, comandos, entornos, estilos y t�tulo de p�gina.
\input{A1.Preambulo.paquetes.tex}
\input{A2.Preambulo.commandos.tex}
\input{A3.Preambulo.entornos.tex}
\input{A4.Preambulo.estilodepagina.tex}

%Comando para crear glosario (index en ingl�s)
\makeindex


% Document body.
%-------------------------------------------------------------------%
\begin{document}

% Formato de documento hasta el cap�tulo 1 (sin incluirlo)
\frontmatter

\renewcommand{\pfcauthorname}{JOS� ANTONIO Y�BENES G�LVEZ}
\renewcommand{\pfctitlename}{Plataforma inal\'{a}mbrica para la web f�sica}
\renewcommand{\pfctutorname}{Ignacio Herrero y Jos� Manuel Cano}
\renewcommand{\pfcanno}{2017}

% Portada.
%%% Tipos de portada %%%
%	1. \maketitle
%	2. titlepage
%%%%%%%%%%%%%%%%%%%%%%%%

%	1. \maketitle
%%%%%%%%%%%%%%%%%%%%%%%%
%\maketitle
%\thispagestyle{empty}

%	2. titlepage
%%%%%%%%%%%%%%%%%%%%%%%%
\begin{titlepage}
	\begin{center}
		\Large{
		UNIVERSIDAD DE M�LAGA
		}
	\end{center}
	\begin{center}
		\Large {
		ESCUELA T�CNICA SUPERIOR DE \\
		INGENIER�A DE TELECOMUNICACI�N
		}
	\end{center}
	
	\vspace{30mm}
	
	\begin{center}\Large{TRABAJO DE FIN DE GRADO}\end{center}
	
	\vspace{20mm}
	
	\begin{center}
		\huge
		\sffamily\scshape
		\pfctitlename		
	\end{center}
	
	\vspace{20mm}
	
	\begin{center}
		\large
		\scshape%\bfseries
		\textsf{GRADO EN INGENIER�A DE \\ TECNOLOG�AS DE TELECOMUNICACI�N}
	\end{center}
	
	\vfill
	
	 \hfill \pfcauthorname  
	 
	 
	 \hfill M�LAGA, \pfcanno 
	 
	\blankpage
	
\end{titlepage}


% Hoja de calificaci�n
\input{B2.Calificacion.tex}

% Resumen del proyecto (formulario)
%%%%%%%%%%%%%%%%%%%%%%%%%%%%%%%%%%
% P�gina de resumen del proyecto %
%%%%%%%%%%%%%%%%%%%%%%%%%%%%%%%%%%

\thispagestyle{empty}
\begin{center}
	\Large \sffamily
	Universidad de M�laga\\
	Escuela T�cnica Superior de Ingenier�a de\\
	Telecomunicaci�n
\end{center}

\bigskip

\begin{center}
	\Huge\scshape
	\pfctitlename
\end{center}

\bigskip

\begin{center}
	\textbf{Autor}\\
	\textsf{\pfcauthorname}
\end{center}

\medskip

\begin{center}
	\textbf{Tutores}\\
	\textsf{\pfctutorname}
\end{center}

\bigskip

\begin{minipage}{\textwidth}
\textbf{Departamento:} Tecnolog�a Electr�nica (DTE)

\medskip

\textbf{Titulaci�n:} Grado en Ingenier�a de Tecnolog�as de Telecomunicaci�n

\medskip

\textbf{Palabras clave:} IOT, web f�sica, web, red, sensores, inal�mbrica, internet, servidor


\medskip

\textbf{Resumen:}\\

Este trabajo de fin de grado implementa la integraci�n de una red inal�mbrica en la banda de 868 MHz utilizando el microcontrolador CC1350 que permite la interacci�n con dispositivos m�viles usando tecnolog�a web, a esto se le conoce como web f�sica.\\

Este proyecto ha implicado el desarrollo de la red inal�mbrica y un servicio web donde es posible monitorizar y gestionar la red. El servicio web consta de un Back-end usando NodeJS y un Front-end utilizando AngularJS.\\

Con el objetivo de demostrar el funcionamiento de la plataforma se ha realizado una aplicaci�n con dos nodos diferentes uno que env�a paquetes de la web f�sica y otro que simular�a que es un aparcamiento de bicicletas p�blicas.

\end{minipage}

\blankpage

% Dedicatoria.

\cleardoublepage
\thispagestyle{empty} % No queremos mostrar ning�n encabezamiento ni pie de p�gina.

\begin{minipage}[c][\textheight][c]{\textwidth} %[pos][height][inner-pos]{width}
\raggedleft %\flushleft

A mi familia, \\
por permitirme cumplir mis objetivos.\\
A Luc�a, \\ por aguantarme.

\bigskip

\emph{El autor}

\end{minipage}

\blankpage

% Acr�nimos

%Lista de acr�nimos 
\chapterbeginx{Acr�nimos}

\begin{acronym}[DLMS/COSEMM]
	%A
	\acro{API}{\textit{Application Programming Interface}}
	\acro{APL}{\textit{Application}}
	%B
	\acro{BLE}{\textit{Bluetooth Low Energy}}
	%C
	\acro{CCS}{\textit{Code Composer Studio}}
	%D
	%E
	\acro{ETSIT}{Escuela T�cnica Superior de Ingenier�a de Telecomunicaci�n}
	%F
	%G
	%H
	\acro{H2M}{\textit{Human-to-Machine}}
	\acro{HP}{\textit{Hewlett-Packard}}
	%I
	\acro{IEEE}{\textit{Institute of Electrical and Electronics Engineers}}
	\acro{IOT}{Internet de las cosas o \textit{Internet Of Things}}
	\acro{ISM}{\textit{Industrial, Scientific and Medical}}
	%J
	%K
	%L
	%M
	\acro{M2M}{\textit{Machine-to-Machine}}
	\acro{MAC}{\textit{Medium Access Control}}
	%N
	\acro{NWK}{\textit{Network}}
	%O
	%P
	\acro{PHY}{\textit{Physical}}
	%Q
	\acro{QR}{\textit{Quick Response}}
	%R
	\acro{RF}{Radiofrecuencia}
	%S
	%T
	\acro{TFG}{Trabajo Fin de grado}
	%U
	\acro{UMA}{Universidad de M�laga}
	\acro{URL}{\textit{Uniform Resource Locator}}
	%V
	%W
	\acro{Wavenis-OSA}{Wavenis Open Standard Alliance}
	\acro{WSN}{\textit{Wireless Sensor Networks}}
	%Y
	%Z	
\end{acronym}

\chapterend

% Tabla de contenidos, figuras y tablas.
\input{B5.TableOfContents.tex}
\input{B5.TableOfFigures.tex}
\input{B5.TableOfTables.tex}

% Formato de documento durante los cap�tulos.
\mainmatter

% Pr�logo.
%%%%%%%%%%%%%%%%%%%%%%%%%%%%%%%%%%%%%%%%%%%%%%%%%%%%%%%%%%%%%%%%%%%%
%%% Documento LaTeX 																						%%%
%%%%%%%%%%%%%%%%%%%%%%%%%%%%%%%%%%%%%%%%%%%%%%%%%%%%%%%%%%%%%%%%%%%
% T�tulo: Pr�logo
% Autor:  Jos� Antonio Y�benes G�vez	
% Fecha:  2016/2017
%%%%%%%%%%%%%%%%%%%%%%%%%%%%%%%%%%%%%%%%%%%%%%%%%%%%%%%%%%%%%%%%%%%

\chapterbeginx{Pr�logo}

El trabajo de fin de grado presentado a continuaci�n lleva el t�tulo de <<\pfctitlename>>. Este trabajo ha sido escrito como parte de los requisitos de graduaci�n para el grado de ingenier�a en tecnolog�as de telecomunicaci�n. El periodo de investigaci�n y redacci�n de este trabajo de fin de grado ha durado desde marzo hasta diciembre de 2017.\\

El tema del trabajo fue formulado conjuntamente con mis tutores, Ignacio Herrero y Jos� Manuel Cano. El proceso de investigaci�n y desarrollo ha sido tedioso, pero realizar un estudio exhaustivo me ha permitido completar el trabajo. Afortunadamente, mis tutores I. Herrero y J.M. Cano, siempre han  estado disponibles y dispuestos a ayudare con todas mis cuestiones.\\

Me gustar�a, por tanto dar las gracias a mis tutores por su excelente orientaci�n y soporte durante todo el proceso e realizaci�n de mi trabajo.\\

Y tambi�n me ha ayudado discutir sobre varios asuntos de mi TFG con mis amigos y familia. Si alguna vez perd� el inter�s, vosotros me mantuvisteis motivado.

	
\chapterend

% Parte I. Introducci�n
\part{Introducci�n}

\chapterbegin{Objetivos}
\label{chp:objetivo}


La web f�sica es un t�rmino que describe la forma de interactuar con cualquier objeto usando la web. A partir este enfoque, es posible navegar y controlar objetos en el mundo a trav�s de dispositivos m�viles. Esto ofrece a los usuarios la forma de realizar sus tareas diarias utilizando los objetos de su entorno. Para utilizar este enfoque, lo primero es seleccionar la manera con la que el objeto se comunicar� con el usuario, tal como c�digos QR o etiquetas RFID. De los diferentes enfoques que tiene la web f�sica, el que se va a tratar en este proyecto es el basado en proximidad inal�mbrica.\\

Con el modelo que plantea la web f�sica los objetos pueden necesitan un canal por donde enviar y recibir datos y el problema se agrava cuando los objetos est�n distribuidos y no tienen cerca un punto de acceso a internet o tienen que funcionar con bater�as. En estos casos, buscar una soluci�n de comunicaci�n inal�mbrica de gran alcance y bajo consumo es prioritario.\\

Con estos conceptos en mente, el presente trabajo de fin de grado va a abordar el concepto de web f�sica, implementando una red inal�mbrica que permita comunicarse con los objetos con el fin de obtener una red vers�til. 

\section{Aplicaci�n de ejemplo}

Para comprobar el funcionamiento de la red, se realizar� una aplicaci�n de ejemplo que podr�a simular un sistema de gesti�n de una ciudad inteligente donde habr�a dos tipos de nodos uno para solo enviar informaci�n a los usuarios (nodo gen�rico) y otro para gestionar un aparcamiento de bicicletas p�blicas (nodo aparcamiento). \\


\subsection{Nodo gen�rico}

El nodo gen�rico es el nodo m�s simple de la plataforma, este permite enviar mensajes de la web f�sica que ser�n definidos desde la web, adem�s este nodo enviar� peri�dicamente mensajes de seguimiento al servidor con informaci�n del estado del dispositivo, como por ejemplo temperatura y bater�a.

\subsection{Nodo aparcamiento}

Este nodo hereda todas las funciones del nodo gen�rico pero adem�s simular� un uso que podr�a tener la web f�sica. En este caso simular� un aparcamiento de bicicletas p�blicas de los que se pueden encontrar por muchas ciudades (v�ase figura \ref{fig:bici-por-malaga}). El dispositivo simular� los estados de los candados que mantiene a cada bicicleta anclada con un array de unos y ceros, donde el uno significar� abierto y el cero significar� cerrado.\\


\begin{figure}[h]
	\centering
	\includegraphics[width=0.8\textwidth]{graphs/bici-por-malaga}
	\caption{Aparcamiento de bicicletas p�blicas de la ciudad de M�laga}
	\label{fig:bici-por-malaga}
\end{figure}

La idea de funcionamiento ser�a simple: el usuario se aproxima a las bicicletas, y le llega una notificaci�n de la web f�sica con un enlace a una web donde poder elegir una bicicleta. Al seleccionar una bicicleta se enviar�a un mensaje al nodo a trav�s de la red inal�mbrica con el comando de abrir el candado de la bicicleta seleccionado.\\

En este proyecto solamente se ha abordado la parte de poder enviar la informaci�n desde la web al nodo, dejando lo dem�s como l�nea futura de este \ac{TFG}.


\chapterend{}
\chapterbegin{Estado del arte}
\label{chp:estarte}
\minitoc
	
\sectionx{Web F\'{i}sica}

\sectionx{Redes inal\'{a}mbricas e internet de las cosas}


\chapterend{}
\chapterbegin{Estructura del documento}
\label{chp:estructura}
%\minitoc




\chapterend{}

\chapterbegin{Metodolog�a y directrices seguidas}
\label{chp:metod}



\chapterend

\part{Desarrollo del proyecto}

% Cap�tulo 01.

\chapterbegin{Descripci�n general}
\label{chp:descripcionGeneral}
\minitoc

\sectionx{Esquema General}

\begin{figure}[h]
	\centering
	\def\svgwidth{0.8\columnwidth}
	\resizebox{0.8\textwidth}{!}{\input{graphs/diagramaGeneral.pdf_tex}}
	\caption{Esquema general del proyecto}
	\label{fig:esquemaGeneral}
\end{figure}
El presente trabajo de fin de grado se puede representar como se observa en la figura \ref{fig:esquemaGeneral}, donde est�n representados los diferentes elementos que componen la arquitectura.\\

Del diagrama se extrae que hay cuatro dispositivos distintos: nodo, concentrador, Raspberry Pi y servidor. Cada uno de estos elementos tiene su propia linea de ejecuci�n diferente de los dem�s. Tambi�n se observa como est�n interconectados los diferentes dispositivos usando protocolos diferentes seg�n sea la comunicaci�n. En las siguientes secciones se repasa la funci�n de cada uno de los dispositivos y como se comunican con los dem�s.

\sectionx{Nodo}

El nodo es el extremos de la red, y es el encargado de comunicarle al usuario la \ac{URL} tal y como define la web f�sica. Este est� basado en el microcontrolador CC1350 SimpleLink\TM de Texas Instruments que nos permite comunicaci�n en dos bandas de frecuencia diferentes, en nuestro caso 868MHz y 2.4Ghz.\\

Con el uso de estas dos bandas de frecuencia, el nodo se comunicar� con el usuario usando la banda de 2.4GHz y el protocolo bluetooth. Y con el resto de la red usando la banda de 868MHz y el protocolo TI 15.4 Stack.\\

\sectionx{Concentrador}

El concentrador es el nodo central de la red TI 15.4 Stack, este se encarga de comunicarse con los nodos a 868MHz. Este dispositivo ejecuta un c�digo precompilado, que implementa una capa 802.15.4e/g MAC/PHY y proporciona una interfaz basada en el protocolo \ac{MT} que conecta el dispositivo con el host linux, en nuestro caso una Raspberry Pi.\\

El concentrador est� compuesto por dos dispositivos diferentes, un dispositivo CC1350 que act�a como coprocesador de una RaspberryPi.


\sectionx{Servidor}

El servidor ofrece una interfaz donde es posible administrar la red y enviar comandos a los sensores. Est� compuesto por un servidor NodeJS y una aplicaci�n AngularJS que dan soporte al \textit{Back-end} y \textit{Fron-end}.\\



\chapterend{}

% Cap�tulo 02.

\chapterbegin{Red Inal�mbrica}
\label{chp:Utiliz}
\minitoc

\sectionx{Nodo}

\sectionx{Protocolo}

\sectionx{Colector}

\chapterend{}

% Cap�tulo 03.

\chapterbegin{Servidor}
\label{chp:App}
\minitoc

\sectionx{Backend}

\sectionx{Frontend}


\chapterend{}

\part{Pruebas y funcionamiento}

\part{Conclusiones y lineas futuras}
\chapterbeginx{Conclusiones}

Despu�s del desarrollo del proyecto, es pertinente hacer una valoraci�n final del mismo, respecto a los resultados obtenidos, las expectativas o el resultado de la experiencia acumulada.\\

Este \ac{TFG} nace con el objetivo de ser un trabajo multidisciplinar que abarque gran parte de los conocimientos adquiridos durante una titulaci�n generalista, como son las comunicaciones, telem�tica y electr�nica. Desde este punto se hizo un an�lisis de diferentes tecnolog�as recientes y se seleccionaron dos como principales: el concepto de ``Web f�sica'' y el microntrolador ``CC1350''. De esta forma surge el objetivo principal de este \ac{TFG} que es la integraci�n de estas dos tecnolog�as en una plataforma que pudiera utilizarse como una soluci�n al problema de la interacci�n humano-m�quina.\\

Durante el desarrollo de este \ac{TFG} surgieron problemas debidos principalmente al peque�o tiempo que llevaba el microcontrolador en el mercado que dificultaba la b�squeda de informaci�n. Con el tiempo fue surgiendo m�s documentaci�n que hizo cambiar la linea de dise�o del proyecto hasta llegar a la que plasma este \ac{TFG}.\\


Finalmente, se han visto concluidos satisfactoriamente los principales objetivos de este proyecto, que han permitido un mejor conocimiento de las tecnolog�as utilizadas y de la metodolog�a de trabajo para llevar un proyecto desde concepto hasta prototipo, pasando por las fases de investigaci�n,   de un producto como ha sido este trabajo de fin de grado.

\chapterend
\chapterbeginx{L�neas futuras}

Durante el desarrollo del \ac{TFG}  se ha observado un crecimiento en el uso de las tecnolog�as que implementa este proyecto, lo que vaticina un futuro donde el concepto de web f�sica sea familiar a la poblaci�n. De este crecimiento se extraen lineas futuras de desarrollo en este �mbito como son:

\begin{itemize}
	\item Mejora del alcance analizando las diferentes alternativas que ofrece TI 15.4-Stack.
	\item Crear un nodo que emule una m�quina expendedora, donde el usuario pueda realizar su compra a trav�s de la web f�sica.
	\item Mejora de la duraci�n de la bater�a optimizando el c�digo y a�adiendo tecnolog�a de recolecci�n de energ�a.
	\item Creaci�n de placas de circuito impreso para nodo y concentrador que permita reducir el tama�o.
\end{itemize}


\begin{flushright}
{\large \pfcauthorname}\nli
\today
\end{flushright}
	
\chapterend

% Anexos
\part{Ap�ndices}

%\appendix

\pagestyle{fancy}
\fancyhead[LE,RO]{\thepage}
\fancyhead[RE]{Ap�ndice} %
\fancyhead[LO]{\nouppercase{\rightmark}}
%\fancyhead[RE]{Parte \thepart \rightmark} %

\chapter{Estructuras de mensajes OTA}
\label{apd:estructurasMensajes}


\begin{figure}[H]
	\centering
	\includegraphics[width=\textwidth]{graphs/mensajesSMSGS}
	\caption{Estructuras de los mensajes del fichero smsgs.h}
	\label{fig:mensajesSMSGS}
\end{figure}


%\pagestyle{fancy}
\fancyhead[LE,RO]{\thepage}
\fancyhead[RE]{Ap�ndice} %
\fancyhead[LO]{\nouppercase{\rightmark}}
%\fancyhead[RE]{Parte \thepart \rightmark} %

\chapter{Ap�ndice}

\minitoc

\section{Primera secci�n}


\chapterend

%\input{D3.AppendixC.tex}

% Formato de documento en la parte final.
\backmatter
%Hace que los cap�tulos y t�tulos nivel inferior no aparezcan numerados (lo que es ideal para conclusiones o notas finales).

% Bibliograf�a
%%%%%%%%%%%%%%%%%%%%%%%%%%%%%%%%%%%%%%%%%%%%%%%%%%%%%%%%%%%%%%%%%%%
%%% Documento LaTeX 																						%%%
%%%%%%%%%%%%%%%%%%%%%%%%%%%%%%%%%%%%%%%%%%%%%%%%%%%%%%%%%%%%%%%%%%%
% T�tulo:		Bibliograf�a
% Autor:  	Jos� Antonio Y�benes G�lvez
% Fecha:  	2017-10-11
% Versi�n:	0.5.0
%%%%%%%%%%%%%%%%%%%%%%%%%%%%%%%%%%%%%%%%%%%%%%%%%%%%%%%%%%%%%%%%%%%%

% Encabezamiento %
\pagestyle{fancy}
\fancyhead[LE,RO]{\thepage}
\fancyhead[LO]{Bibliograf�a}
%\fancyhead[RE]{Parte \thepart \rightmark} %
\fancyhead[RE]{\nouppercase{\rightmark}} %

%Inclusi�n de bibliograf�a%
\bibliography{E2.Bibliografia} %�sese el nombre del fichero sin extensi�n

%Inclusi�n en el �ndice (Tabla de contenidos)
\addcontentsline{toc}{chapter}{Bibliograf�a}

%Formateo de estilo de bibliograf�a
% Otros formatos: plain, unsrt, abbrv
%  plain: las entradas se ordenan alfab�ticamente y se etiquetan con un n�mero (p.ej., [1])
% unsrt: igual que plain, pero aparecen en orden de citaci�n.
% alpha: el etiquetado se hace por autor y a�o de publicaci�n (p.ej., [Knu66]).
% abbrv: igual que alpha, pero m�s abreviado.
% spain: igual que plain, siguiendo las referencias del "Diccionario de ortogrf�a t�cnica" de J.Mart�nez de Sousa
% IEEEannotMod: normativa IEEE espa�olizada por mi
\bibliographystyle{unsrt}

%Impresi�n de todas las entradas bibliogr�ficas
\nocite{*}

\chapterend

% �ndice alfab�tico%
%\input{F1.Index.tex}

\end{document}