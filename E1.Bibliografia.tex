%%%%%%%%%%%%%%%%%%%%%%%%%%%%%%%%%%%%%%%%%%%%%%%%%%%%%%%%%%%%%%%%%%%
%%% Documento LaTeX 																						%%%
%%%%%%%%%%%%%%%%%%%%%%%%%%%%%%%%%%%%%%%%%%%%%%%%%%%%%%%%%%%%%%%%%%%
% T�tulo:		Bibliograf�a
% Autor:  	Jos� Antonio Y�benes G�lvez
% Fecha:  	2017-10-11
% Versi�n:	0.5.0
%%%%%%%%%%%%%%%%%%%%%%%%%%%%%%%%%%%%%%%%%%%%%%%%%%%%%%%%%%%%%%%%%%%%

% Encabezamiento %
\pagestyle{fancy}
\fancyhead[LE,RO]{\thepage}
\fancyhead[LO]{Bibliograf�a}
%\fancyhead[RE]{Parte \thepart \rightmark} %
\fancyhead[RE]{\nouppercase{\rightmark}} %

%Inclusi�n de bibliograf�a%
\bibliography{E2.Bibliografia} %�sese el nombre del fichero sin extensi�n

%Inclusi�n en el �ndice (Tabla de contenidos)
\addcontentsline{toc}{chapter}{Bibliograf�a}

%Formateo de estilo de bibliograf�a
% Otros formatos: plain, unsrt, abbrv
%  plain: las entradas se ordenan alfab�ticamente y se etiquetan con un n�mero (p.ej., [1])
% unsrt: igual que plain, pero aparecen en orden de citaci�n.
% alpha: el etiquetado se hace por autor y a�o de publicaci�n (p.ej., [Knu66]).
% abbrv: igual que alpha, pero m�s abreviado.
% spain: igual que plain, siguiendo las referencias del "Diccionario de ortogrf�a t�cnica" de J.Mart�nez de Sousa
% IEEEannotMod: normativa IEEE espa�olizada por mi
\bibliographystyle{IEEEannotMod}

%Impresi�n de todas las entradas bibliogr�ficas
\nocite{*}

\chapterend