%%%%%%%%%%%%%%%%%%%%%%%%%%%%%%%%%%
% P�gina de resumen del proyecto (Ingles) %
%%%%%%%%%%%%%%%%%%%%%%%%%%%%%%%%%%

\thispagestyle{empty}
\begin{center}
	\Large \sffamily
	Universidad de M�laga\\
	Escuela T�cnica Superior de Ingenier�a de\\
	Telecomunicaci�n
\end{center}

\bigskip

\begin{center}
	\Huge\scshape
	\pfctitlenameenglish
\end{center}

\bigskip

\begin{center}
	\textbf{Author}\\
	\textsf{\pfcauthorname}
\end{center}

\medskip

\begin{center}
	\textbf{Supervisors}\\
	\textsf{\pfctutorname}
\end{center}

\bigskip

\begin{minipage}{\textwidth}
\textbf{Department:} Tecnolog�a Electr�nica (DTE)

\medskip
\textbf{Degree:} Grado en Ingenier�a de Tecnolog�as de Telecomunicaci�n

\medskip

\textbf{Keywords:} IOT, physical web, web, network, sensors, wireless, physical web, internet

\medskip

\textbf{Abstract:}\\

In this final degree project, several CC1350 microcontroller are used to create a wireless network operating in the 868 MHz band. These microcontrollers can additionally broadcast Bluetooth Low Energy advertisement packets, thus allowing the interaction with mobile hand-held devices usign web technology. This communication paradigm is known as the physical web.\\

In addition to the deployment of the wireless network, a web service has been built in order to monitor and manage it. The web service consists of a back-end module based on NodeJS and a front-end application developed with AngularjS.\\

In order to verifty the performance of the platform, a test application with two different nodes has been created. One of the nodes  sends packets from the physical web and the other one is responsible for simulating a public bicycle parking.


\end{minipage}

\blankpage