%%%%%%%%%%%%%%%%%%%%%%%%%%%%%%%%%%
% P�gina de resumen del proyecto (Ingles) %
%%%%%%%%%%%%%%%%%%%%%%%%%%%%%%%%%%

\thispagestyle{empty}
\begin{center}
	\Large \sffamily
	Universidad de M�laga\\
	Escuela T�cnica Superior de Ingenier�a de\\
	Telecomunicaci�n
\end{center}

\bigskip

\begin{center}
	\Huge\scshape
	\pfctitlenameenglish
\end{center}

\bigskip

\begin{center}
	\textbf{Author}\\
	\textsf{\pfcauthorname}
\end{center}

\medskip

\begin{center}
	\textbf{Supervisors}\\
	\textsf{\pfctutorname}
\end{center}

\bigskip

\begin{minipage}{\textwidth}
\textbf{Department:} Tecnolog�a Electr�nica (DTE)

\medskip
\textbf{Degree:} Grado en Ingenier�a de Tecnolog�as de Telecomunicaci�n

\medskip

\textbf{Keywords:} IOT, physical web, web, network, sensors, wireless, physical web, internet

\medskip

\textbf{Abstract:}\\

This final degree project implements the integration of a wireless network in the 868 MHz band usign the CC1350 microcontroller, which allows the interaction with mobile devices usign web technology, this is known as the physical web.\\

This project has involved the devolopment of the wireless network and a web service where is possible to monitor and manage the own network. The web service consists of a back-end application usign NodeJS and a front-end application usign AngularjS.\\

In order to prove the performance of the platform, an application has been made with two different nodes, one that sends packets from the physical web and another one which is responsible of simulating a public bicycle parking.

\end{minipage}

\blankpage