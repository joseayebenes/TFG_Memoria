
\chapterbegin{Descripci�n general del proyecto}
\label{chp:ti15stack}
\minitoc
\begin{figure}[h]
	\centering
	\def\svgwidth{\columnwidth}
	\resizebox{\textwidth}{!}{\input{graphs/diagramaGeneral.pdf_tex}}
	\caption{Esquema general del proyecto}
	\label{fig:esquemaGeneral}
\end{figure}

\sectionx{Visi�n General}

Este proyecto se presenta c�mo una soluci�n al problema de comunicarse desde tel�fonos m�viles con otros objetos, para ello se crea una red inal�mbrica que permite la comunicaci�n con un nodo de la red a trav�s de una web sin la necesidad de aplicaciones m�viles extra.\\

Durante el desarrollo de este \ac{TFG} se resuelven dos problemas concretos usando esta tecnolog�a: un aparcamiento de bicicletas y una baliza que env�a informaci�n a los usuarios.\\

En la figura \ref{fig:esquemaGeneral} se refleja como ser�a el esquema de red utilizado. Esta red se compone de 5 elementos diferenciados que se comunican de distintas formas entre ellos. En este cap�tulo procederemos a explicar la funci�n de cada dispositivo en la red.

\sectionx{Tel�fono m�vil}

El tel�fono m�vil es la interfaz del usuario y ser�a el encargado de recibir la informaci�n que le env�a el nodo como un paquete bluetooth del tipo Eddystone URL enmarcado en la tecnolog�a de la web f�sica. Para que el usuario pueda recibir este paquete necesita un dispositivo con bluetooth �4.0? y activar la opci�n de "Web f�sica" en el navegador Chrome. \\

Con esta configuraci�n cada vez que el usuario est� al alcance de un nodo, recibir� una notificaci�n con la \ac{URL}.

\sectionx{Nodo}

En el extremo de la red se encuentran los nodos, que su principal cometido es enviar paquetes bluetooth a los tel�fonos m�viles adem�s de ejecutar las ordenes que se le env�an desde la web.\\

Para comunicarse con la red, utiliza el protocolo propietario Ti802.15 de Texas Instruments.


\sectionx{Colector}


\sectionx{Servidor}



\chapterend